\documentclass{beamer}
\usepackage{graphicx} % Required for inserting images

\usepackage[utf8]{inputenc}
\usepackage[T1]{fontenc}
\usepackage{lmodern}
\usepackage{amsmath,amssymb}
\usepackage{microtype}
\usepackage{ellipsis}
%\usepackage[ngerman]{babel}
\let\openbox\undefined
\usepackage{mathtools}
\usepackage{enumitem}
\let\openbox\undefined
\usepackage{amsthm}
\usepackage{thmtools}
\usepackage{graphicx}
\usepackage{stmaryrd}
\usepackage{tikz}
\usetikzlibrary{positioning}
\usepackage{algpseudocode}
\usepackage[absolute,overlay]{textpos}
\usepackage{url}
\usepackage[
backend=biber,
style=numeric,
]{biblatex}
\addbibresource{references.bib}
\usepackage[normalem]{ulem}
\usepackage{verbatim}
\usepackage{subcaption} % allow for subfigures

\usepackage[ruled, algosection]{algorithm2e}


\declaretheoremstyle[
  spaceabove=\parsep,
  spacebelow=0,
  headfont=\bfseries,
  notefont=\bfseries,
  notebraces={(}{)},
  bodyfont=\normalfont,
  postheadspace=.5em
]{definition}

\newtheoremstyle{plain}         % name
    {\parsep}                   % Space above
    {}                          % Space below
    {\itshape}                  % Body font
    {}                          % Indent amount
    {\bfseries}                 % Theorem head font
    {.}                         % Punctuation after theorem head
    {.5em}                      % Space after theorem head
    {\thmname{#1}\thmnumber{ #2}\thmnote{ \bfseries (#3)}}                 % Theorem head spec (can be left empty, meaning ‘normal’)

\theoremstyle{plain}

% \declaretheorem[sharenumber=algocf]{theorem}
% \declaretheorem[sharenumber=algocf]{lemma}
% \declaretheorem[sharenumber=algocf]{corollary}
\declaretheorem[sharenumber=algocf]{proposition}

% \declaretheorem[sharenumber=algocf]{definition}
% \declaretheorem[sharenumber=algocf]{example}
\declaretheorem[sharenumber=algocf]{remark}
\declaretheorem[sharenumber=algocf]{notation}

\renewcommand\qedsymbol{$\square$}

\newcommand\R{\mathbb R}
\newcommand\Z{\mathbb Z}
\newcommand\N{\mathbb N}
\newcommand\C{\mathbb C}
\newcommand{\Q}{\mathbb Q}
\newcommand{\F}{\mathbb{F}}
\newcommand{\ass}{\underline{Assume:}  }
\newcommand{\zz}{\underline{t.s.:}  }

\renewcommand{\phi}{\varphi}
\renewcommand{\epsilon}{\varepsilon}

\usetheme[compress]{Berlin}
\setbeamertemplate{footline}[frame number]{}
\setbeamertemplate{navigation symbols}{}
\setbeamertemplate{footline}{}

\makeatletter
\beamer@theme@subsectionfalse%
\makeatother


\title{Computing fundamental domains of crystallographic groups}
\subtitle{With connections to topological interlocking}
\author{Lukas Schnelle}
\date{GAPDays Summer 2024}

\begin{document}

% remove dots on first slide
\frame[plain]{\titlepage}

\section{Definitions}

% Fund. Domain
% Crystallographic group
% Bieberbach
% volume of fund dom
% Dirichlet Cells
\begin{frame}
    \begin{definition}\label{def:isometry}
        Let $\phi:\R^n \to \R^n$ be a surjective map. 
        Then $\phi$ is called an \emph{isometry} if: \pause
        $$
            \forall v, w \in \R^n : \pause d(v^\phi, w^\phi) = d(v,w).
        $$
        with $d(-,-)$ the Euclidean distance.\\ \pause 
        The set of all isometries of dimension $n$ is denoted as $E(n)$ and called the \emph{Euclidean group}.
    \end{definition}

    \begin{lemma}\label{lma:isom-is-grp}
        Let $E(n)$ be the set of all isometries of a dimension $n \in \N$. \\ \pause
        Then $E(n)$ is a group with the composition of homomorphisms as the group operation.
    \end{lemma}
\end{frame}

\begin{frame}
    \begin{proposition}[{{\cite[Exa. 1.1, Prop. 1.6]{szczepanski2012geometry}}}]
        There is an isometry
        $$
            E(n) \cong O(n) \ltimes \R^n.
        $$
        We denote with $\phi_o$ the orthogonal part of $\phi$ and with $\phi_t$ the vector/translation part of $\phi$.
    \end{proposition}

    Then the group operation of $\phi, \psi \in E(n)$ is as follows:\pause
    $$
        (\phi_o, \phi_t) \circ (\psi_o, \psi_t) \coloneqq (\underbrace{\phi_o \circ \psi_o}_{\substack{\text{op. in }O(n)\\ \text{ i.e. comp. of maps}}}, \psi_t^{\phi_o} + \phi_t),
    $$\pause
    and the action of $E(n)$ on $\R^n$ extends to the action of $O(n) \ltimes \R^n$ on $\R^n$:\pause
    \begin{align*} \label{align:semidirect-action}
        \R^n \times ( O(n) \ltimes \R^n) \to \R^n: (v, (\phi_o, \phi_t)) \mapsto v^{(\phi_o, \phi_t)} = v^{\phi_o} + \phi_t.
    \end{align*}
\end{frame}

\begin{frame}
    \begin{definition}\label{def:system-of-reps}
        Let $\lambda$ be a partition of $\R^n$ and let $\emptyset \neq V \subseteq \R^n$ be a set.\\ \pause
        Then we call $V$ a \emph{system of representatives} of the partition $\lambda$ if $V$ contains exactly one element of each class of $\lambda$.
    \end{definition} \pause

    \begin{definition}\label{def:fund-dom}
        Let $\Gamma \leq E(n)$ be a subgroup and $F \subseteq \R^n$ a closed set.
        Then $F$ is called a \emph{fundamental domain for $\Gamma$} if:\pause
        \begin{enumerate}[label=(\roman*)]
            \item $\bigcup_{\gamma \in \Gamma} F^{\langle \gamma \rangle} = \R^n$, \pause
            \item there is a system of representatives $V \subseteq \R^n$ w.r.t.\ the partition given by the orbits of $\Gamma$ acting on $\R^n$ such that $$F^\circ \subseteq V \subseteq F.$$
        \end{enumerate}
    \end{definition}
\end{frame}

\begin{frame}
    the problem of Computing Dirichlet Cells
\end{frame}

\begin{frame}
    current approach to computations
        - theorem that word length corresponds to distance 
        - algo that incorporates that knowledge
\end{frame}

\begin{frame}
    connections to TIA -> deformations
\end{frame}

\begin{frame}
    \textbf{\Large Thank you for your attention}\\ 
    \bigskip
    References:\\
    \printbibliography
\end{frame}

\end{document}
